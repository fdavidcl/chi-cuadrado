%%
% Modificación de una plantilla de Latex para adaptarla al castellano.
%%

%%%%%%%%%%%%%%%%%%%%%
% Thin Sectioned Essay
% LaTeX Template
% Version 1.0 (3/8/13)
%
% This template has been downloaded from:
% http://www.LaTeXTemplates.com
%
% Original Author:
% Nicolas Diaz (nsdiaz@uc.cl) with extensive modifications by:
% Vel (vel@latextemplates.com)
%
% License:
% CC BY-NC-SA 3.0 (http://creativecommons.org/licenses/by-nc-sa/3.0/)
%
%%%%%%%%%%%%%%%%%%%%%

%----------------------------------------------------------------------------------------
%	PACKAGES AND OTHER DOCUMENT CONFIGURATIONS
%----------------------------------------------------------------------------------------

\documentclass[a4paper, 10pt]{article} % Font size (can be 10pt, 11pt or 12pt) and paper size (remove a4paper for US letter paper)
%\usepackage{helvet}
%\renewcommand{\familydefault}{\sfdefault}
\usepackage[protrusion=true,expansion=true]{microtype} % Better typography
\usepackage{graphicx} % Required for including pictures
\usepackage[usenames,dvipsnames]{color} % Coloring code
\usepackage{wrapfig} % Allows in-line images
\usepackage[utf8]{inputenc}
\usepackage{enumerate}
\usepackage{enumitem}

% Imágenes
\usepackage{graphicx}

\usepackage{amsmath}
% para importar svg
%\usepackage[generate=all]{svgfig}

% sudo apt-get install texlive-lang-spanish
\usepackage[spanish]{babel} % English language/hyphenation
\selectlanguage{spanish}
% Hay que pelearse con babel-spanish para el alineamiento del punto decimal
\decimalpoint
\usepackage{dcolumn}
\newcolumntype{d}[1]{D{.}{\esperiod}{#1}}
\makeatletter
\addto\shorthandsspanish{\let\esperiod\es@period@code}
\makeatother

\usepackage{longtable}
\usepackage{tabu}
\usepackage{supertabular}

\usepackage{multicol}
\newsavebox\ltmcbox

% Para algoritmos
\usepackage{algorithm}
\usepackage{algorithmic}
\usepackage{amsthm}

% Para matrices
\usepackage{amsmath}

% Símbolos matemáticos
\usepackage{amssymb}
\usepackage{accents}
\let\oldemptyset\emptyset
\let\emptyset\varnothing

\usepackage[hidelinks]{hyperref}

\usepackage[section]{placeins} % Para gráficas en su sección.
\usepackage[T1]{fontenc} % Required for accented characters
\usepackage{tikz}
\newenvironment{allintypewriter}{\ttfamily}{\par}
\setlength{\parindent}{0pt}
\parskip=8pt
\linespread{1.05} % Change line spacing here, Palatino benefits from a slight increase by default

\makeatletter
\renewcommand\@biblabel[1]{\textbf{#1.}} % Change the square brackets for each bibliography item from '[1]' to '1.'
\renewcommand{\@listI}{\itemsep=0pt} % Reduce the space between items in the itemize and enumerate environments and the bibliography
\newcommand{\figura}[2]{\begin{figure}[hbtp]\centering \includegraphics[width=90mm]{#1} \caption{#2} \end{figure}}

\renewcommand{\maketitle}{ % Customize the title - do not edit title and author name here, see the TITLE block below
\begin{center} % Center align
{\Huge\@title} % Increase the font size of the title
\end{center}

\vspace{20pt} % Some vertical space between the title and author name

\begin{flushright} % Right align
{\large\@author} % Author name
\\\@date % Date

\vspace{40pt} % Some vertical space between the author block and abstract
\end{flushright}
\renewcommand{\baselinestretch}{0.5}

}
%----------------------------------------------------------------------------------------
%	TITLE
%----------------------------------------------------------------------------------------

\title{\textbf{Distribución $\chi$-cuadrado}\\ % Title
\vspace{20 pt}
} % Subtitle

\author{\textsc{Daniel López\\
David Charte} % Author
\\{\textit{Universidad de Granada}}} % Institution

\date{\today} % Date

%----------------------------------------------------------------------------------------
%\setcounter{secnumdepth}{3}
\usepackage{anysize}
\marginsize{3cm}{3cm}{2.5cm}{2.5cm}

\newtheorem{theorem}{Teorema}[section]
\newtheorem{definition}{Definición}[section]
\newtheorem{property}{Propiedad}[section]


\begin{document}
\maketitle
\tableofcontents
\setcounter{page}{1}
\pagebreak

\section{Definición}

La distribución $\chi$-cuadrado se puede definir como la suma de cuadrados de variables aleatorias siguiendo distribuciones normales 0-1:

\begin{definition}[Distribución $\chi$-cuadrado]
  Si $X_1,X_2,\dots X_p$ son variables aleatorias independientes con distribución $N(0,1)$, entonces a la distribución que sigue $\chi^2=X_1^2+X_2^2+\dots X_p^2$ la llamamos $\chi$-cuadrado con $p$ grados de libertad.
\end{definition}

\subsection{Función de densidad}
Probaremos que la distribución $\chi$-cuadrado es un caso particular de la distribución Gamma. Recordamos la función de densidad de esta distribución:

\begin{definition}[Distribución Gamma]
  Decimos que la variable aleatoria $X$ sigue una distribución Gamma si su función de densidad es:
  $$f(x\mid \alpha, \beta) = \frac 1 {\Gamma(\alpha)\beta^\alpha}x^{\alpha-1}e^{-\frac x \beta},\quad x\in[0,+\infty[,\quad \alpha,\beta>0~.$$
\end{definition}

Recordamos además el siguiente teorema:
\begin{theorem}
  Sean X e Y variables aleatorias \textbf{continuas} con funciones generatrices de momentos $M_X(t)$ y $M_Y(t)$ respectivamente. Además, supongamos que las funciones \textbf{existen} para $t$ en un entorno de 0 y que son continuas en $t=0$. Entonces, si $M_X(t)=M_Y(t)~\forall t$ las variables aleatorias tienen la misma función de densidad.
\end{theorem}

Sea $\chi^2=X_1^2+X_2^2+\dots X_p^2$ una variable aleatoria con distribución $\chi$-cuadrado, y sea $Y$ una variable aleatoria siguiendo una distribución $\Gamma(\alpha, \beta)$. Calculamos las funciones generatrices de momentos. Por un lado, la de $\chi^2$ se expresará como:
\begin{align*}
  M_{\chi^2}(t)
  &=E\left[e^{t\chi^2}\right]=E\left[e^{t(X_1^2+X_2^2+\dots+X_p^2)}\right]\\
  &=E\left[e^{tX_1^2}\right]E\left[e^{tX_2^2}\right]\dots E\left[e^{tX_p^2}\right]
    =E\left[e^{tX_1^2}\right]^p
\end{align*}
puesto que las normales son independientes. Calculamos ahora $E\left[e^{tX_1^2}\right]$:
\begin{align*}
  E\left[e^{tX_1^2}\right]
  &=\int_{-\infty}^\infty e^{x^2t}\frac{e^{-x^2/2}}{\sqrt{2\pi}} dx
   =\int_{-\infty}^\infty\frac{\mathrm{exp}\left(-\frac{x^2}{2(1-2t)^{-1}}\right)}{\sqrt{2\pi}} dx\\
  &=\frac 1 {\sqrt{1-2t}}
    \int_{-\infty}^\infty\frac{\mathrm{exp}\left(-\frac{x^2}{2(1-2t)^{-1}}\right)}{\sqrt{(1-2t)^{-1}}\sqrt{2\pi}} dx\\
  &\left[\textit{La integral vale 1 por ser la función de densidad de }N(0,(1-2t)^{-1})\textit{ en su dominio}\right]\\
  &=\frac 1 {\sqrt{1-2t}}
\end{align*}

Por tanto, $M_{\chi^2}(t)=\frac 1 {(\sqrt{1-2t})^p}$.

Ahora calculamos la función generatriz de momentos de la variable $Y$:
\begin{align*}
  M_Y(t)&=E\left[e^{tY}\right]=\int_0^\infty e^{ty} \frac 1 {\Gamma(\alpha)\beta^\alpha}y^{\alpha-1}e^{-\frac y \beta} dy\\
  &=\int_0^\infty \frac{\left(\frac 1 \beta\right)^\alpha y^{\alpha-1}e^{-y\left(\frac 1 \beta - t\right)}}{\Gamma(\alpha)} dy\\
  &= \left(\frac{\frac 1 \beta}{\frac 1 \beta - t}\right)^\alpha\frac 1 {\Gamma(\alpha)}
     \int_0^\infty \left(\frac 1 \beta - t\right)\left(\left(\frac 1 \beta - t\right)y\right)^{\alpha-1}e^{-y\left(\frac 1 \beta - t\right)} dy\\
  \left[z=\left(\frac 1 \beta - t\right)y\right] &= \left(\frac{\frac 1 \beta}{\frac 1 \beta - t}\right)^\alpha\frac 1 {\Gamma(\alpha)}
     \int_0^\infty z^{\alpha-1}e^{-z} dz=
     \left(\frac{\frac 1 \beta}{\frac 1 \beta - t}\right)^\alpha=\frac 1 {(1 -\beta t)^\alpha}
\end{align*}

Si evaluamos la expresión obtenida en $\alpha=\frac p 2,~\beta=2$ obtenemos el mismo resultado que en la función que nos da la distribución $\chi$-cuadrado. Por tanto, la variable $\chi^2$ se distribuye según una $\Gamma(\frac p 2, 2)=:\chi^2(p)$.

Obtenemos de esta forma la siguiente definición equivalente de la distribución $\chi$-cuadrado:
\begin{definition}[Distribución $\chi$-cuadrado]
  Decimos que la variable aleatoria $X$ sigue una distribución $\chi$-cuadrado con $p$ grados de libertad si su función de densidad es:
  $$f(x\mid p) = \frac 1 {\Gamma(\frac p 2)2^{\frac p 2}}x^{\frac p 2-1}e^{-\frac x 2},\quad x\in[0,+\infty[,\quad p\in\{1,2,\dots\}~.$$
\end{definition}

\figura{pdf.png}{Función de densidad de la distribución $\chi$-cuadrado. Imagen de Wikipedia (CC BY).}

\section{Función generatriz de momentos}
% La función generatriz de momentos se define como sigue:
% $$\phi(t)=E[e^{-tX}]=\sum_{x=0}^{\infty}e^{-t x}\frac{e^{-\theta}\theta^x}{x!}=e^{-\theta}\sum_{x=0}^{\infty}e^{t x}\frac{\theta^x}{x!}=e^{-\theta}\sum_{x=0}^{\infty}\frac{(\theta e^t)^x}{x!}=e^{-\theta}e^{\theta e^t}=e^{\theta(e^t-1)}$$
$$M_X(t)=\left(\frac 1 {1 - 2t}\right)^{\frac p 2},\quad t<\frac 1 2$$

\subsection{Esperanza}
% Para el cálculo de la esperanza tan solo debemos derivar la función generatriz de momentos una vez y evaluarla con t=0:
% $$\frac{\partial\phi}{\partial t} = \theta e^t e^{\theta(e^t-1)}$$
% Evaluamos la expresión en t=0 y tenemos que:
$$\mathrm E[X]=p$$

\subsection{Varianza}
% La varianza se puede expresar como $\sigma^2=E[X^2]-E[X]^2$ por lo que es tan solo calcular el momento de orden 2 usando la función generatriz de momentos, por lo que volvemos a derivar la expresión anterior:
% $$\frac{\partial^2\phi}{\partial t^2} = \theta e^t e^{\theta(e^t-1)}+\theta^2 e^{2t} e^{\theta(e^t-1)}$$
% Evaluamos la expresión en t=0 y tenemos que:
% $$E[X^2]=\theta+\theta^2$$
% Y por lo tanto la expresión de la varianza es la siguiente:
$$\mathrm{Var}[X]=2p$$

\section{Inferencia}

\subsection{Información de Fisher}

\begin{definition}[Información de Fisher]
  Para una variable $X$ cuya distribución tiene función de densidad $f(x\mid \theta)$, la información de Fisher se define como
  $$\mathrm I_X(\theta)=\mathrm E_\theta\left[\left(\frac \partial {\partial \theta} \log f(x\mid \theta)\right)^2\right]$$
\end{definition}

Recordamos que, bajo ciertas circunstancias, la información de Fisher se puede calcular mediante la siguiente expresión:
$$\mathrm I_X(\theta)=\mathrm E_\theta\left[-\frac {\partial^2} {\partial \theta^2} \log f(x\mid \theta)\right]$$

En el caso de la distribución $\chi$-cuadrado, calculamos primero la derivada:
\begin{align*}
  \frac\partial {\partial p} \log f(x\mid p)
  &= \frac\partial {\partial p} \log\left(\frac 1 {\Gamma(\alpha)\beta^\alpha}x^{\alpha-1}e^{-\frac x \beta}\right)
   = \frac\partial {\partial p}\left(\left(\frac p 2 - 1\right)\log x - \frac x 2 - \log\Gamma\left(\frac p 2\right) - \frac p 2 \log 2\right)\\
  &= \frac 1 2 \log x - \frac 1 2 \psi\left(\frac p 2\right) - \frac 1 2 \log 2 = \frac 1 2\left(\log \frac x 2 - \psi\left(\frac p 2\right)\right)
\end{align*}
\begin{align*}
  \frac {\partial^2} {\partial p^2} \log f(x\mid p)
  &= \frac\partial {\partial p} \frac 1 2\left(\log \frac x 2 - \psi\left(\frac p 2\right)\right) = -\frac 1 4 \psi'\left(\frac p 2\right)
\end{align*}

Por tanto, la información de Fisher nos queda:
\begin{align*}
  \mathrm I_X(p)
  &=\mathrm E_p\left[-\frac {\partial^2} {\partial p^2} \log f(x\mid p)\right] = \int \left(-\frac {\partial^2} {\partial p^2} \log f(x\mid p)\right)f(x\mid p)~\mathrm dx\\
  &=\int \frac 1 4 \psi'\left(\frac p 2\right)f(x\mid p)~\mathrm dx
   =\frac 1 4 \psi'\left(\frac p 2\right)\int f(x\mid p)~\mathrm dx
   =\frac 1 4 \psi'\left(\frac p 2\right)
\end{align*}

\section{Referencias}
\begin{enumerate}
  \item Notes on the Chi-squared Distribution - Georgia Tech Institute: \url{http://people.math.gatech.edu/~ecroot/3225/chisquare.pdf}
  \item Probability. An Introduction - Geoffrey Grimmet
  \item Statistical Inference - George Casella, Roger Berger
\end{enumerate}

\end{document}
